\section*{\large РЕФЕРАТ}

Отчет 26 с., 8 рис., 4 табл., 13 источн. 

ОБУЧЕНИЕ С ПОДКРЕПЛЕНИЕМ, ГЛУБОКОЕ ОБУЧЕНИЕ, ТЕОРИЯ ИГР

Объектом исследования являются алгоритмы обучения с подкреплением в среде с несколькими агентами. 

\textbf{Цель данной научно-исследовательской работы} --- провести анализ существующих методов обучения с подкреплением в среде с несколькими агентами (MARL).

Для достижения поставленной цели необходимо решить следующие задачи:
\begin{itemize}[label=---]
	\item формализовать задачу используя математический аппарат;
	\item провести анализ предметной области методов обучения с подкреплением для задач игрового искусственного интеллекта;
	\item сформулировать способы классификации методов;
	\item классифицировать методы исходя из способов;
	\item сравнить описанные алгоритмы;
	\item отразить результаты сравнения и классификации рассмотренных алгоритмов в выводе.
\end{itemize}

Результатом работы является выявление критериев применимости алгоритмов к задаче игорового искусственного интеллекта.

 \pagenumbering{gobble}
