\section*{\large РЕФЕРАТ}

Отчет 26 с., 8 рис., 4 табл., 13 источн. 

ОБУЧЕНИЕ С ПОДКРЕПЛЕНИЕМ, ГЛУБОКОЕ ОБУЧЕНИЕ, ТЕОРИЯ ИГР

Хуй пойми, что это за среда, но это там, где агенты друг с другом поиграть могут. Игра может быть как в кооперацию (командная игра), так и в конкуренцию (один на всех), а иногда еще и смешанный вариант (куча команд и все друг против друга).

Наша научно-исследовательская работа посвящена методам, как можно наебать наседающих собесов в таких средах. Мы хотим разобраться во всех существующих методах обучения с подкреплением для игр с несколькими агентами (вроде бы такое название МАРЛ).

Чтобы понять, как же всё это говно работает, мы должны выполнить несколько задач:
\begin{itemize}[label=---]
\item задача номер один - сделать эту хуету формальной, каким-то математическим языком;
\item потом нам нужно изучить все методы обучения с подкреплением для игр, чтобы понять, что же это за беда такая;
\item после этого мы должны придумать какую-то классификацию для методов;
\item ну и тогда можно будет уже разобраться, какие методы куда относятся и чем они отличаются;
\item а потом можно будет еще эти методы друг с другом сравнить и посмотреть, что из этого всего хуже, а что лучше;
\item в итоге мы все это говно сведем воедино и напишем какой-то вывод, чтобы кто-то потом мог этим пользоваться.
\end{itemize}

Результатом нашей хуйни будет то, что мы разберемся, когда можно использовать какие алгоритмы в играх, а когда лучше даже не пробовать, ибо получишь жопу вместо золота.

 \pagenumbering{gobble}
