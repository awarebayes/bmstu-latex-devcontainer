%----------------------- Преамбула -----------------------

\usepackage{extsizes} % Для добавления в параметры класса документа 14pt

% Для работы с несколькими языками и шрифтом Times New Roman по-умолчанию
\usepackage[english,russian]{babel}
\usepackage{fontspec}
\setmainfont{Times New Roman}
\usepackage[left=30mm,right=10mm,top=20mm,bottom=20mm]{geometry}
\usepackage{misccorr}
\usepackage{indentfirst}
\usepackage{enumitem}
\setlength{\parindent}{1.25cm}
\usepackage{multirow}
\renewcommand{\baselinestretch}{1.5}
\setlist{nolistsep} % Отсутствие отступов между элементами \enumerate и \itemize

% Дополнительное окружения для подписей
\usepackage{array}
\newenvironment{signstabular}[1][1]{
	\renewcommand*{\arraystretch}{#1}
	\tabular
}{
	\endtabular
}

% Переопределение стандартных \section, \subsection, \subsubsection по ГОСТу;
% Переопределение их отступов до и после для 1.5 интервала во всем документе
\usepackage{titlesec}

% \filcenter
%\titleformat{\section}[block]
%{\bfseries\normalsize}{\thesection}{1em}{}
\titlespacing{\section}{0mm}{0mm}{8mm}

\titleformat{\subsection}[hang]
{\bfseries\normalsize}{\thesubsection}{1em}{}
\titlespacing\subsection{\parindent}{12mm}{0mm}

\titleformat{\subsubsection}[hang]
{\bfseries\normalsize}{\thesubsubsection}{1em}{}
\titlespacing\subsubsection{\parindent}{12mm}{0mm}

\titleformat{name=\section}[block]
{\normalfont\normalsize\bfseries\hspace{\parindent}}
{\thesection}
{1em}
{}
\titleformat{name=\section,numberless}[block]
{\normalfont\normalsize\bfseries\centering}
{}
{0pt}
{}

\titlespacing\section{0mm}{0mm}{0mm}
\titlespacing\chapter{0mm}{-12mm}{0mm}

\newcommand{\anonsection}[1]{%
	\section*{\centering#1}%
	\addcontentsline{toc}{section}{#1}%
}

\newcommand{\specsection}[1]{
	\section*{\centering#1}
	\addcontentsline{toc}{section}{#1}
}

% Работа с изображениями и таблицами; переопределение названий по ГОСТу
\usepackage{caption}
\captionsetup[figure]{name={Рисунок},labelsep=endash}
\captionsetup[table]{singlelinecheck=false, labelsep=endash}

\usepackage{graphicx}
\usepackage{diagbox} % Диагональное разделение первой ячейки в таблицах

% Цвета для гиперссылок и листингов
\usepackage{color}

% Гиперссылки \toc с кликабельностью
\usepackage[linktoc=all]{hyperref}
\hypersetup{hidelinks}

% Листинги
%\setsansfont{Arial}
%\setmonofont{Courier New}

\usepackage{color}
\hypersetup{citecolor=black}

\usepackage{listings}
% Для листинга кода:
\lstset{ %
	basicstyle=\small\ttfamily,			% размер и начертание шрифта для подсветки кода
	numbers=none,						% где поставить нумерацию строк (слева\справа)
	%numberstyle=,					% размер шрифта для номеров строк
	%stepnumber=1,						% размер шага между двумя номерами строк
	%numbersep=5pt,						% как далеко отстоят номера строк от подсвечиваемого кода
	frame=single,						% рисовать рамку вокруг кода
	tabsize=4,							% размер табуляции по умолчанию равен 4 пробелам
	captionpos=t,						% позиция заголовка вверху [t] или внизу [b]
	breaklines=true,					
	breakatwhitespace=true,				% переносить строки только если есть пробел
	backgroundcolor=\color{white},
	showstringspaces=false,
	extendedchars=true,
	escapeinside=§§
}

\DeclareCaptionLabelSeparator{line}{\ --\ }
\captionsetup[lstlisting]{
	singlelinecheck=false,
	margin=0pt,
	labelsep=line
}

\usepackage{ulem} % Нормальное нижнее подчеркивание
\usepackage{hhline} % Двойная горизонтальная линия в таблицах
\usepackage[figure,table]{totalcount} % Подсчет изображений, таблиц
\usepackage{rotating} % Поворот изображения вместе с названием
\usepackage{lastpage} % Для подсчета числа страниц

\makeatletter
\renewcommand\@biblabel[1]{#1\hfill}
\makeatother

\usepackage{color}
\usepackage[cache=false, newfloat]{minted}
\newenvironment{code}{\captionsetup{type=listing}}{}
\SetupFloatingEnvironment{listing}{name=Листинг}

\usepackage{amsmath}
\usepackage{slashbox}
\titleformat{\chapter}{\large\bfseries}{\thechapter}{12mm}{\large\bfseries}



\usepackage{subcaption}
\usepackage{float}
\usepackage{booktabs}
