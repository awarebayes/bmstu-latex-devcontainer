\chapter*{\hfill{\centering ЗАКЛЮЧЕНИЕ }\hfill}
\addcontentsline{toc}{chapter}{ЗАКЛЮЧЕНИЕ}
Че касается работы, то мы сделали кучу дел, типа:

\begin{itemize}[label=---]
\item задача была обосрана с помощью игры маркова;
\item провели анализ темы, чтобы понять о чем эта хуета;
\item смекнули, как можно было бы разобраться с методами;
\item отсортировали методы по нашим критериям;
\item сравнили алгоритмы, чтоб понять какой из них больше нравится;
\item выписали свои нахуй выводы из сравнения алгоритмов.
\end{itemize}

Модные алгоритмы классического обучения, типа IPPO, IA2C, работают нормально, когда кругом куча народу, а вот если всего пару уебанов, то централизованная Q--сеть гораздо лучше работает. Актер-критик тоже может быть полезен, когда надо работать с большими и непрерывными действиями.

Но если действия у нас дискретные, как в этой залупе, то лучше использовать алгоритмы без обучения поощрением, типа MAVEN или IQL. Если вдруг действий много, то MAVEN еще лучше, потому что он умеет разбираться с средами и показывает лучшую сходимость в них.

