\chapter*{\hfill{\centering ВВЕДЕНИЕ }\hfill}
\addcontentsline{toc}{chapter}{ВВЕДЕНИЕ}

% Введение должно содержать оценку современного состояния решаемой научно-технической проблемы
Задача принятия решений является важно нерешенной задачей во многих областях.
Одним из подходов в задачах принятия решений является обучение с подкреплением.
% основание и исходные данные для разработки темы
Проблема формулируется как Марковский процесс принятия решений. Такая формулировка позволяет использовать определенный подход.
Классические алгоритмы обучения с подкреплением (Reinforcement \newline Learning, RL) подразумевают среду с одним агентом.
В данной работе рассматриваются проблемы, в которых несколько агентов. 
Они должны взаимодействовать друг с другом, при этом взаимодействие может выражаться в кооперации, в конкуренции, или в смешанном варианте.


Примером предметной области применения нескольких агентов являются: управление в робототехнике, менеджмент ресурсов, коллаборативные системы принятия решений, майнинг данных и т. д.
Рассматривается предметная область игрового искусственного интеллекта.
Она достаточно простая в сравнении. Также в ней нет необходимости взаимодействовать с реальным миром, что позволяет использовать симуляцию.

Средой будем называть игровое пространство, в которой происходит взаимодействие агентов.
Среда может подразумевать кооперацию (одна команда), конкуренцию (один против всех), а также смешанную конкуренцию и кооперацию одновременно (несколько команд).

% обоснование необходимости проведения НИР
Классификация алгоритмов обучения в среде с несколькими агентами позволит выбрать наиболее подходящий алгоритм для решения конкретной задачи.

% Во введении промежуточного отчета по этапу НИР должны быть указаны цели и задачи исследований, выполненных на данном этапе, их место в выполнении отчета о НИР в целом
\textbf{Цель данной научно-исследовательской работы} --- провести анализ существующих методов обучения с подкреплением в среде с несколькими агентами (MARL).

Для достижения поставленной цели необходимо решить следующие задачи:
\begin{itemize}[label=---]
	\item формализовать задачу используя математический аппарат;
	\item провести анализ предметной области методов обучения с подкреплением для задач игрового искусственного интеллекта;
	\item сформулировать способы классификации методов;
	\item классифицировать методы исходя из способов;
	\item сравнить описанные алгоритмы;
	\item отразить результаты сравнения и классификации в выводе.
\end{itemize}


