\chapter*{\hfill{\centering ВЕДЕНИЕ }\hfill}
\addcontentsline{toc}{chapter}{ВВЕДЕНИЕ}

% Введение должно содержать оценку современного состояния решаемой научно-технической проблемы

Ёбнутая задача принятия решений - блять, она всё ещё нахуй нерешаемая во многих областях жизни.
Но у нас тут есть один годный подход для ее решения - обучение с подкреплением.

Блять, такая проблема: Марковский процесс принятия решений. Какая-то хуйня, но нам с этим жить. Вот заебали эти классические алгоритмы обучения с подкреплением, они только с одним агентом работают. Мы тут рассматриваем проблемы, где несколько таких хуев. Им надо как-то между собой общаться, типа кооперироваться, конкурировать, или смешивать это все в одну хуйню. Но вот хуй знает, как это все делается.

% основание и исходные данные для разработки темы
Братишки, нахуй, смотри, когда несколько хулиганов пытаются вместе что-то сделать, то это может быть применено в разных областях: от управления роботами до управления нашими дураками. Одна из таких областей - игровой искусственный интеллект. В отличие от других областей, в игровом искусственном интеллекте все просто, но в то же время удобно. Не нужно иметь дело с реальным миром, и можно использовать симуляцию. Это позволяет нашим головастым пацанам изобрести новые хитрости, а также обучить их взаимодействию друг с другом.

% обоснование необходимости проведения НИР
Эй, ребята, чекайте, ща мы говорим про среду, где происходит взаимодействие агентов. Среда может быть как кооперативной, так и конкурентной, а то и смешанной, где несколько команд боьют друг другому ебасосы. Нам короче нужно обосновать, зачем проводить эту научно-исследовательскую работу. Мы знаем, что классификация алгоритмов обучения в среде с несколькими агентами может помочь выбрать самый подходящий алгоритм для решения нашей задачи. Понятно, друзья?


% Во введении промежуточного отчета по этапу НИР должны быть указаны цели и задачи исследований, выполненных на данном этапе, их место в выполнении отчета о НИР в целом

\textbf{Это наше говнососное исследование} --- нащупать, какие хуевые методы обучения с подкреплением лучше всего въебывают в играх с несколькими игроками (MARL).

Чтобы сделать эту блядскую научку по обучению с подкреплением для игрух, надо сначала ее задачи хуй пойми как матаном описать:
\begin{itemize}[label=---]
	\item определить, что за нахуй надо сделать;
	\item понять, какие хуйни с подкреплением подходят для игрух;
	\item придумать, как ебануть методы в какие-то сраные ящики;
	\item распихать все эти ящики по категориям;
	\item побороться, кто из них круче;
	\item написать в итоге, что заебись получилось.
\end{itemize}